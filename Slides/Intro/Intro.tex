\documentclass{beamer}
\setbeamertemplate{navigation symbols}{}
\usepackage[latin1]{inputenc}
\usepackage{setspace,dsfont}
\usepackage{amsmath,amssymb,pdfpages}
\usepackage[longnamesfirst,nonamebreak]{natbib}
\usepackage[english]{babel}
\usepackage{eurosym,multirow,hyperref,cmll}
\usepackage{listings}

\newcommand{\Lik}{\mathcal{L}}
\newcommand{\lau}{\lambda_u}
\newcommand{\wi}{\underline{w}}
\newcommand{\m}{\mathcal{M}}
\newcommand{\wa}{\overline{w}}
\newcommand{\lae}{\lambda_e}
\newcommand{\1}{\mathbb{1}}
\newcommand{\F}{\mathcal{F}}
\newcommand{\D}{\mathcal{D}}
\newcommand{\f}{\mathfrak{f}}
\newcommand{\E}{\mathbb{E}}
\newcommand{\V}{\mathbb{V}}
\newcommand{\N}{\mathbb{N}}
\newcommand{\Real}{\mathbb{R}}
\newcommand{\X}{\mathcal{X}}
\newcommand{\A}{\mathcal{A}}
\newcommand{\B}{\mathcal{B}}
\newcommand{\ow}{\overline{w}}
\newcommand{\uw}{\underline{w}}
\newcommand{\de}{\delta}

\hypersetup{
    colorlinks,%
    citecolor=blue,%
    filecolor=blue,%
    linkcolor=blue,%
    urlcolor=blue,
}
%\usetheme{Boadilla}
%\usetheme{Marburg}
%\usetheme{Hannover}
%\usetheme{Pittsburgh}
%\usetheme{umbc1}
%\usetheme{Montpellier}
\usetheme{Singapore}
%\usetheme{}
%\usetheme{}
%\lstset {language=C++}

\DeclareMathOperator{\plim}{plim}
\DeclareMathAlphabet{\mathpzc}{OT1}{pzc}{m}{it}

\beamersetuncovermixins{\opaqueness<1>{25}}{\opaqueness<2->{15}}
\begin{document}
\begin{frame}
\title{Computational Methods in Economics}
\subtitle{Introduction}
\titlepage
\end{frame}

\section{Generalities}

\frame{\frametitle{Objectives}
\begin{itemize}
\item Introduction to serious data work. 
\item Numerical methods to solve complex problems.
\item Computational implementation. 
\item Use data, model, and estimation 
\end{itemize}
}

\frame{\frametitle{Example}
\begin{itemize}
\item Consider an optimal stopping model
\begin{itemize}
 \item Search for a product
 \item Search for a job
\end{itemize}
\item Key feature is a reservation value. 
\end{itemize}
}

\frame{\frametitle{Basic job search theory (1) - The basic model}
\begin{itemize}
\item Job search theory arises initially out of a basic model describing the behavior of a person looking for work in a situation of imperfect information
\item The basic job search model has the following assumptions
\begin{itemize}
\item They are not allowed to select the intensity of their search 
\item They cannot look for jobs once they are employed
\item They cannot recall job offers once rejected (sequential search)
\end{itemize}
\end{itemize}
}
\frame{\frametitle{Basic job search theory (2) - The basic model}
\begin{itemize}
\item The job-seeker does not know exactly what wage each job pays. So by looking, he can expect to improve his prospect of earnings.
\item We further assume that this distribution is the same at each date, and that successive wage offers are independent draws from this distribution
\item The optimal strategy of a person looking for work consists simply of choosing a \emph{reservation wage} that represents the lowest remuneration he will accept
\end{itemize}
}

\begin{frame}\frametitle{Wage posting}
Revisiting \\ Burdett and Mortensen(1998) \\ Bontemps, Robin and Van der berg(2000)
\end{frame}

\begin{frame}\frametitle{Assumptions}
\begin{itemize}
\item Population $m$
\item Stock of unemployed $u$
\item Employment opportunities occur at rate $\lau$ representing the parameter of a Poisson Distribution. 
\item Employment are destroyed at rate $\de$
\item The utility of an agent consists of the wage $w$ if she is employed, or $b$ if she is unemployed. 
\item The distributions of offered and accepted wages are respectively denoted by $F$, and $G$ with support $[\uw,\ow]$
\item Individuals discount future earnings at rate $\rho$
\item Firms post contract. 
\end{itemize}
\end{frame}

\begin{frame}\frametitle{Agents problem: with on-the-job-search}
Unemployed agents
\begin{equation*}
\rho V_u = b + \lau \int_{\phi}^{\overline{w}} \big[V^e(x) -V^u\big] dF(x)  
\end{equation*}

Employed agents
\begin{equation*}
\rho V_e(w) = w + \de[ V^u - V^e(w) ] + \lae \int_{w}^{\overline{w}} \big[V^e(x) -V^e(w)\big] dF(x)  
\end{equation*}
\end{frame}

\begin{frame}\frametitle{Reservation wage (1)}
The reservation wage is given by: 
\begin{eqnarray*}
\phi 
 &=& b + \big[\lau -\lae\big] \int_{\phi}^{\overline{w}} \frac{\overline{F}(x)}{\rho+\delta+\lambda_e\overline{F}(x)}d(x) 
\end{eqnarray*}
\end{frame}

\begin{frame}\frametitle{Reservation wage (2)}
This is nonlinear equation, which requires root-finding techniques!
\end{frame}


\begin{frame}\frametitle{Reservation wage (3)}
This is nonlinear equation, and as the wage offer distribution $F()$ is not observed, evaluating the reservation wage requires numerical integration. 
\end{frame}

\begin{frame}\frametitle{Reservation wage (4)}
Assume b has a distribution such that there exists a $\phi(b)$. Evaluating $\phi$ is costly, so one may be tempted to create a grid for $b$, and evaluate for the values of the grid, and then use interpolation techniques to recover $\phi(b)$.
\end{frame}

\begin{frame}\frametitle{Reservation wage (5)}
\begin{itemize}
\item Given data on wages, it is possible to estimate all the structural parameters $\lau,\lae,\delta$ and the parameters of the wage offer distribution. 
\item Let $\Theta$ be the set of parameters to be estimated and then consider the criterion function
 \begin{equation}
  min_{\Theta} || W - \hat{\phi}(\Theta) ||
 \end{equation}
\item Nonlinear optimization. 
\end{itemize}
\end{frame}


\frame{\frametitle{Organization}
\begin{itemize}
\item Tuesday: Lecture
\item Thursday: Applications
\item Office hours: Email appointment 
\item Questions? 
\end{itemize}
}

\frame{\frametitle{Evaluation}
\begin{itemize}
\item Class applications (40\%)
\item Problem sets (60\%)
\end{itemize}
}

\frame{\frametitle{Plan of the course}
\begin{itemize}
\item Algorithm: Theory and Introduction to R/C++
\item Data: Theory and Application
\item Maximum Likelihood Estimation
\item Root Finding Techniques
\item Numerical Optimization
\item Simulation Techniques
\item Numerical Integration
\item Interpolation and Extrapolation
\item Dynamic Programming
\item Final
\end{itemize}
}

\frame{\frametitle{R/C++}
 \begin{itemize}
  \item Build R. 
  \item Install Rstudio. 
  \item Install some packages (Rcpp, RcppArmadillo, Zelig).
  \item Compile a first c++ file. 
 \end{itemize}
}



\end{document}


